\documentclass[11pt]{extarticle}
\usepackage{fullpage}
\usepackage[ampersand]{easylist}
\ListProperties(Hide=10, Style*=$\bullet\;\,$, Style2*=$\;\,${\tiny$\blacksquare$}$\;\,$,Space*=1mm,Space2*=0.1mm)

\usepackage{hyperref}
\hypersetup{colorlinks=true,
	linkcolor=black,
	filecolor=black,      
	urlcolor=black}

\usepackage{amsmath,amssymb,amsthm,mathtools,mathrsfs}
\newtheorem{thm}{Theorem}[]
\usepackage{bibref}

\usepackage[T1]{fontenc}
\usepackage[sc,osf]{mathpazo}
\usepackage{eulervm}
\usepackage[bold=.05]{xfakebold}

\usepackage{tikz}
\usetikzlibrary{calc}
\usetikzlibrary{shapes}
\usepackage{pgfplots}
\pgfplotsset{trig format plots=rad}
\usetikzlibrary {3d}
\pgfplotsset{compat=1.18}

\usepackage[most]{tcolorbox}
\tcbuselibrary{skins}
\usepackage[explicit]{titlesec}
\newtcolorbox{secbox}[1][]{enhanced,attach boxed title to top center,drop fuzzy shadow,breakable,colbacktitle=gray,colback=black,colframe=black,
	coltext=white,size=title,title={#1}}
\titleformat{\section}[runin]{\bfseries\LARGE}{}{0pt}{\hfill
	%\begin{secbox}[\thesection]
	%	\centering #1
	%\end{secbox}
	%}
\tcbsidebyside[sidebyside adapt=left,segmentation style=solid,enhanced,size=small]
{%
	\thesection 
}
{%
	#1
}
}
\titleformat{\subsection}[runin]{\bfseries\large}{}{0pt}
{\hfill
%	\begin{secbox}[\thesubsection]
	%		\centering #1
	%	\end{secbox}
\tcbsidebyside[sidebyside adapt=left,segmentation style=solid,enhanced,size=small]
{%
	\thesubsection 
}
{%
	#1
}
}
\titleformat{\subsubsection}[runin]{\bfseries}{}{0pt}
{\hfill
%		\begin{secbox}[\thesubsubsection]
	%			\centering #1
	%		\end{secbox}
\tcbsidebyside[sidebyside adapt=left,segmentation style=solid,enhanced,size=small]
{%
	\thesubsubsection 
}
{%
	#1
}
}

\usepackage{multicol}
\setlength{\columnsep}{5mm}
\setlength\columnseprule{.1pt}

\newcommand{\ra}{\rightarrow}
\newcommand{\R}{\mathbb{R}}
\newcommand{\C}{\mathbb{C}}
\newcommand{\Na}{\mathbb{N}}
\newcommand{\Z}{\mathbb{Z}}
\newcommand{\Q}{\mathbb{Q}}
\newcommand{\w}[1]{\text{#1}}
\newcommand{\ck}{.\,.\,}
\newcommand{\sm}[2]{\displaystyle\sum_{#1}^{#2}}
\newcommand{\Uint}[2]{\overline{\int\!}_{#1}^{\;#2}}
\newcommand{\Lint}[2]{\underline{\int\!}_{\;#1}^{\;#2}}
\newcommand{\pfrac}[2]{\frac{\partial#1}{\partial#2}}
\newcommand{\ckfil}{$.\dotfill.$}
\newcommand{\tm}{\times}
\newcommand{\snote}[1]{{\footnotesize(#1)}}
\newcommand{\st}{\,{}_{s}|_t\,}
\newcommand{\gen}[1]{\langle #1 \rangle}
\newcommand{\tbx}[2][]{
\begin{tcolorbox}[enhanced,breakable,size=small,colback=black!2!white,title={#1},arc is angular, arc=1.5mm,drop fuzzy shadow]
	#2
\end{tcolorbox}
}
\newcommand{\y}{$\blacksquare\;$}

\author{Yashas.N}
\title{Numerical Linear Algebra}
\date{}
\begin{document}
\maketitle
	\begin{multicols}{2}
		\tbx[]{if a matrix is in triangular form one can easy calculate its inverse by making note that inverse of a triangular 
			matrix is of the same triangular type i.e. for example $ A $ is upper triangular non-singular matrix then 
			$ A^{-1} $ is also an upper triangular matrix.}
			\tbx[$LU$ Decomposition]{
				\y From previous point we have if any non singular matrix $ A $ can be written as $ A=LU $ for lower triangular $ L $ and upper triangular $ U $ then $ A^{-1}=U^{-1}L^{-1} $ thus inverse can be easily calculated.\\  
				\y This Decomposition may not be unique\\
				\y To decompose in a easy way we take diagonal elements of $ U $ or $ L $ as $ 1. $ (only in one of the factors) and compute the coefficients by writing $ A=LU $ and solving some equations  in a linear order.\\
				\y Now in addition if principal minors $ (\Delta_k) $ of matrix $ A $ are not zero then the above decomposition is unique.}
				\tbx[Gauss elimination]{
				if $ A=[a_{ij}] $ be a $ n\tm n $ non singular matrix then for linear system $ Ax=b$ then we can use elementary operations:\\
				exchange of rows, addition of rows and multiplication by a non zero constant to a row to transform the  linear system$A'x=b'$ such that $ a'_{11}\neq 0  $ and $ a'_{i1}=0 $ for $ i<1 $  and continuing this process to get for $ i= 2,3,\ck, n$ we get a system $ Gx=\tilde{b} $ where $ G $ is upper triangular and has same solutions as original system.}
				\tbx[Gauss-Jordan method]{ this method is similar to Gauss elimination but $ Ax=b $ for non singular square $ A $ is transformed to $ G_Jx=\tilde{b} $ where $ G_J $ is diagonal i.e. for $ A=[a_{ij}] $, $ a_{ii} $ is made non zero and all other $ a_{ij} $ is made zero with elementary transformations.}
			  \tbx[General Iterative methods]{
			  \y iterative methods can be generalised as $ x^{(k)} =Tx^{(k-1)}+c$ \\
			  \y this method converges to a unique solution for any initial approximation $ x^{(0)} $ iff \snote{$\iff$} $ \rho(T)<1$ where $\rho(T)=\max(|\lambda|) $ for $ \lambda $ eigenvalue of $ T. $ }
				\tbx[Jacobi's Method]{\y if $ Ax=b $ is a system such that for $ n $-square $ A=[a_{ij}] $ we have $ a_{ii}\neq 0 $ \snote{if not is made by rearranging rows or equations if possible} then for $ x=[x_i]^T $ we can transform $ x_i=  \left[\sm{\substack{j=1 \\  j\neq i}}{n}(-a_{ij}x_j/a_{ii}) +b_i/a_{ii}\right]$ from which we get the iterative method i.e. $ x^{(0)} $ is initial approximation and for $ k^{th} $ approximation $ x^{(k)} $ we have the iteration using $ x^{(k-1)} $ given by
				\[ x_i^{(k)}=\frac{ 1 }{a_{ii}} \left[\sm{\substack{j=1 \\ j\neq i}}{n}(-a_{ij}x_j^{(k-1)}) +b_i\right].\]
				 \y Now for matrix representation if $ A=D+L+U $ where $ D $ is diagonal $ L $ is lower diagonal with diagonal entries $ 0 $ and $ U $ is upper diagonal with diagonal entries $ 0 $ then for Jacobi method we have 
				 \begin{align*}
				 	(D+L+U)x&=b\\
				 	\implies Dx&= -(L+U)x+b.\\
				 	\implies x&=-D^{-1}(L+U)x+D^{-1}b.\\
				 	i.e. \quad x^{(k)}&=-D^{-1}(L+U)x^{(k-1)}+D^{-1}b.
				 	\end{align*}  
			 	so we get $ T=-D^{-1}(L+U),c=D^{-1}b $ for general form.}
				 \tbx[Gauss-Seidel Method]{ \y This is similar to Gauss method but here we use the previous $ k^{th} $ iterated variables for the next $ k^{th} $ one  i.e. for in $ x_i^{(k)} $ iteration we can replace $ x_j^{(k-1)} $ for $ j<i $ with $ x_j^{(k)} $ as these are already found i.e.\\
				 	 $x_i^{(k)}=$
				 \[\quad \frac{ 1 }{a_{ii}} \left[-\sm{j=1}{i-1}a_{ij}x_j^{(k)}-\sm{j=i+1}{n}a_{ij}x_j^{(k-1)}+b_i\right].\] 
				 \y for matrix representation we rewrite the iterative formula as \\
				 $\sm{j=1}{i}a_{ij}x_j^{(k)}=-\sm{j=i+1}{n}a_{ij}x_j^{(k-1)}+b_i$\\
				 similar to Jacobi's case if $ A=D+L+U $ by above formula we have 
				{ \small
					\begin{align*}
				 	(D+L)x^{(k)}&=-Ux^{(k-1)}+b.\\
				 	i.e. x^{(k)}&= -(D+L)^{-1}Ux^{(k-1)}+(D+L)^{-1}b.
				 	\end{align*}}
			 	so we get $ T=-(D+L)^{-1}U,c=(D+L)^{-1}b. $ }
			 	\tbx{ for system $ Ax=b $ , $ A=D+L+U $ \\
			 		\y if $  A$ is strictly diagonal then both Jacobi and Gauss-Seidel methods converge for every initial approximation $ x^{(0)}. $ \\
			 		\y  Gauss-Seidel method is twice as fast as Jacobi's method for convergence \\
			 		now from general iterative methods we have \\
			 		\y sufficient condition for convergence of Jacobi's method is that 
			 		\[||T||=||-D^{-1}(L+U)||<1\quad i.e.\; \rho(T)<1.\]
			 		\y similarly  sufficient condition for convergence of Gauss-Seidel method is that 
			 		\[ ||T||=||-(D+L)^{-1}U||<1.\]
			 		\y Both these method also converge if $ A =[a_{ij}]$ is such that \\
			 		 $ \sm{\substack{j=1\\ j\neq i}}{n}|a_{ij}|\leq |a_{ii}| \text{ for } i=1,2,\ck,n $ and strict inequality holds for at least one $ i $.   }
			 		 
			 		 
\begin{thebibliography}{9}
	\bibitem{NABF} 
	Burden R. L., Faires D. J., Burden A. M.: Numerical Analysis, Cengage Learning,(2016).
	\bibitem{NASS}
	S. S. Sastry: Introductory Methods of Numerical Analysis, PHI Learning,(2012).
\end{thebibliography}			 		 
			 		 
	\end{multicols}


\end{document}